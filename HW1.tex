%Source for HW1
\documentclass[11pt]{article}
% page size parameters
\usepackage{latexsym,setspace,xspace}
\setlength\itemindent{0.1in}
\setlength{\itemsep}{0in}

\setlength\topmargin{-0.75in}
\setlength\oddsidemargin{0.0in}
\setlength\evensidemargin{0.0in}
\setlength\parindent{0.0in} 
\setlength\textwidth{6.5in}
\setlength\textheight{9.25in}
\setlength{\parskip}{\medskipamount}

\newenvironment{newmath}{\begin{displaymath}%
\setlength{\abovedisplayskip}{4pt}%
\setlength{\belowdisplayskip}{4pt}%
\setlength{\abovedisplayshortskip}{3pt}%
\setlength{\belowdisplayshortskip}{3pt} }{\end{displaymath}}


% header hackery

\newlength{\toppush}
\setlength{\toppush}{2\headheight}
\addtolength{\toppush}{\headsep}


%\def\subjnum{CS 4/585}
%\def\subjname{Cryptography}

%\def\doheading#1#2#3{\vfill\eject\vspace*{-\toppush}%
%\vbox{\hbox to\textwidth{{\bf}\subjnum: \subjname \hfil Handout #1\strut}%
%\hbox to\textwidth{PSU --- Tom Shrimpton\hfil#3\strut}%
 %s   \hrule}}

\newcommand{\htitle}[1]{\vspace*{3.25ex}%
  \begin{center}{\Large\bf #1}\end{center}}


% page style that displays just footers
\makeatletter
\def\ps@justfooters{\let\@mkboth\@gobbletwo\def\@oddhead{}%
	\def\evenhead{}}
\makeatother

% \handout{handout label}{title}{date}

\newcommand{\handout}[3]{\thispagestyle{justfooters}%
  \markboth{\subjnum\ Handout \protect\ref{#1}: #2}%
    {\subjnum\ Handout \protect\ref{#1}: #2}%
  \pagestyle{myheadings}%
  \doheading{\protect\ref{#1}}{#2}{#3}%
  \htitle{#2}}

\newcommand{\procfont}[1]{\textbf{#1}}
\newcommand{\nfont}[1]{{\footnotesize #1}}


\newcommand{\dash}{{\mbox{-}}}
\newcommand{\taglen}{{\tau}}
\newcommand{\Pad}{P}
\newcommand{\Ctr}{{S}}

\newcommand{\outputs}{\Rightarrow}
\newcommand{\prf}{\mathrm{prf}}
\newcommand{\prp}{\mathrm{prp}}
\newcommand{\tprp}{\mathrm{tprp}}
\newcommand{\sprp} {{\pm\mathrm{prp}}}
\newcommand{\stprp}{{\pm\mathrm{tprp}}}
\newcommand{\ind} {{\mathrm{ind\mbox{-}{cpa}}}}
\newcommand{\auth} {{\mathrm{auth}}}
\newcommand{\indr}{{\mathrm{ind\$\mbox{-}{cpa}}}}
\newcommand{\indrcca}{{\mathrm{ind\$\mbox{-}{cca}}}}
\newcommand{\indcca}{{\mathrm{ind\mbox{-}{cca2}}}}
\newcommand{\nmcca}{{\mathrm{nm\mbox{-}cca}}}

\newcommand{\xxx}{{\mathrm{xxx}}}
\newcommand{\Encrypt}{\mathrm{Encrypt}}
\newcommand{\Decrypt}{\mathrm{Decrypt}}

\newcommand{\ERROR}{{\textsc{Error}}}
\newcommand{\invalid}{{\textsc{Invalid}}}

\newcommand{\CBC}{\mathrm{CBC}}
\newcommand{\CTR}{\mathrm{CTR}}
\newcommand{\CT}{{\mathit{CT}}}

\newcommand{\captionfont}{\small\sf }

\newcommand{\sep}{\:}
\newcommand{\kwfont}{\bf}
\newcommand{\AND}{\mbox{\kwfont and}}
\newcommand{\ALGORITHM}{\mbox{\kwfont Algorithm}}
\newcommand{\BEGIN}{\mbox{\kwfont begin}}
\newcommand{\DO}{\mbox{\kwfont do}}
\newcommand{\ELSE}{\mbox{\kwfont else}}
\newcommand{\END}{\mbox{\kwfont end}}
\newcommand{\FOR}{\mbox{\kwfont for}}
\newcommand{\IF}{\mbox{\kwfont if}}
\newcommand{\RETURN}{\mbox{\kwfont return}}
\newcommand{\THEN}{\mbox{\kwfont then}}
\newcommand{\TO}{\mbox{\kwfont to}}

\newcommand{\ZERO}{{\textbf{0}}}
\newcommand{\ONE}{{\textbf{1}}}
\newcommand{\error}{\mbox{\sc error}}
\newcommand{\true}{{\mathtt{true}}}
\newcommand{\false}{{\mathtt{false}}}
\newcommand{\undef}{{\textsc{undefined}}}
\newcommand{\dotdot}{{\,..\,}}
\newcommand{\xor}{{\,\oplus\,}}
\newcommand{\bits}{{\{0,1\}}}
\newcommand{\bitss}{\bits^*}
\newcommand{\e}{{\epsilon}}
\newcommand{\outputlen}{\varsigma}

\newlength{\saveparindent}
\setlength{\saveparindent}{\parindent}
\newlength{\saveparskip}
\setlength{\saveparskip}{\parskip}



% ====================================================================

% Theorem environments

\newtheorem{thm}{Theorem} %[section]
\newtheorem{lem}[thm]{Lemma}
\newtheorem{cor}[thm]{Corollary}
\newtheorem{propo}[thm]{Proposition}
\newtheorem{defn}[thm]{Definition}
\newtheorem{assm}[thm]{Assumption}
\newtheorem{clm}[thm]{Claim}
\newtheorem{rem}[thm]{Remark}

\newenvironment{theorem}{\begin{thm}\begin{sl}}%
{\end{sl}\end{thm}}
\newenvironment{lemma}{\begin{lem}\begin{sl}}%
{\end{sl}\end{lem}}
\newenvironment{corollary}{\begin{cor}\begin{sl}}%
{\end{sl}\end{cor}}
\newenvironment{proposition}{\begin{propo}\begin{sl}}%
{\end{sl}\end{propo}}
\newenvironment{definition}{\begin{defn}\begin{em}}%
{\end{em}\end{defn}}
\newenvironment{assumption}{\begin{assm}\begin{em}}%
{\end{em}\end{assm}}
\newenvironment{claim}{\begin{clm}\begin{sl}}%
{\end{sl}\end{clm}}
\newenvironment{remark}{\begin{rem}\begin{em}}%
{\end{em}\end{rem}}


\newcommand{\secref}[1]{\mbox{Section~\ref{#1}}}
\newcommand{\apref}[1]{\mbox{Appendix~\ref{#1}}}
\newcommand{\thref}[1]{\mbox{Theorem~\ref{#1}}}
\newcommand{\prref}[1]{\mbox{Proposition~\ref{#1}}}
\newcommand{\defref}[1]{\mbox{Definition~\ref{#1}}}
\newcommand{\corref}[1]{\mbox{Corollary~\ref{#1}}}
\newcommand{\lemref}[1]{\mbox{Lemma~\ref{#1}}}
\newcommand{\clref}[1]{\mbox{Claim~\ref{#1}}}
\newcommand{\propref}[1]{\mbox{Proposition~\ref{#1}}}
\newcommand{\figref}[1]{\mbox{Figure~\ref{#1}}}
\newcommand{\eqref}[1]{\mbox{(\ref{#1})}}

\newcommand{\maybespace}{{\ifnum\springer=0{{\ }}\fi}}


\def\sqed{{\hspace{5pt}\rule[-1pt]{3pt}{9pt}}}
% deluxe proof environment
\def\qsym{\vrule width0.6ex height1em depth0ex}
\newcount\proofqeded
\newcount\proofended
\def\qed{ {\hspace{5pt}\rule[-1pt]{3pt}{9pt}}
\end{rm}\addtolength{\parskip}{-0pt}
\setlength{\parindent}{\saveparindent}
\global\advance\proofqeded by 1 }

\newenvironment{proof}%
 {\proofstart}%
 {\ifnum\proofqeded=\proofended\qed\fi \global\advance\proofended by 1
  \medskip}
\makeatletter
\def\proofstart{\@ifnextchar[{\@oprf}{\@nprf}}
\def\@oprf[#1]{\begin{rm}\protect\vspace{6pt}\noindent{\bf Proof of #1:\
}%
\addtolength{\parskip}{5pt}\setlength{\parindent}{0pt}}
\def\@nprf{\begin{rm}\protect\vspace{6pt}\noindent{\bf Proof:\ }%
\addtolength{\parskip}{5pt}\setlength{\parindent}{0pt}}

% ========================================================================

% Lists

\newcounter{ctr}
\newcounter{savectr}
\newcounter{ectr}

\newenvironment{newenum}{%
\begin{list}{{\rm (\arabic{ctr})}\hfill}{\usecounter{ctr} \labelwidth=18pt%
\labelsep=7pt \leftmargin=25pt \topsep=3pt%
\setlength{\listparindent}{\saveparindent}%
\setlength{\parsep}{\saveparskip}%
\setlength{\itemsep}{2pt} }}{\end{list}}

\newenvironment{bignewenum}{%
\begin{list}{{\bf (\arabic{ctr})}\hfill}{\usecounter{ctr} \labelwidth=18pt%
\labelsep=7pt \leftmargin=25pt \topsep=6pt%
\setlength{\listparindent}{\saveparindent}%
\setlength{\parsep}{\saveparskip}%
\setlength{\itemsep}{6pt} }}{\end{list}}

\newenvironment{tiret}{%
\begin{list}{\hspace{2pt}\rule[0.5ex]{6pt}{1pt}\hfill}{\labelwidth=18pt%
\labelsep=7pt \leftmargin=25pt \topsep=3pt%
\setlength{\listparindent}{\saveparindent}%
\setlength{\parsep}{\saveparskip}%
\setlength{\itemsep}{1pt} }}{\end{list}}

\newenvironment{onelist}{%
\begin{list}{{\rm (\arabic{ctr})}\hfill}{\usecounter{ctr} \labelwidth=15pt%
\labelsep=2pt \leftmargin=17pt \topsep=1pt% 
\setlength{\listparindent}{\saveparindent}%
\setlength{\parsep}{\saveparskip}%
\setlength{\itemsep}{0pt} }}{\end{list}} 

\newenvironment{twolist}{%
\begin{list}{{\rm (\arabic{ctr}.\arabic{ectr})}%
\hfill}{\usecounter{ctr} \labelwidth=37pt%
\labelsep=7pt \leftmargin=44pt \topsep=0pt%
\setlength{\listparindent}{\saveparindent}%
\setlength{\parsep}{\saveparskip}%
\setlength{\itemsep}{0pt} }}{\end{list}}

\newenvironment{newitemize}{%
\begin{list}{$\bullet$}{\labelwidth=19pt%
\labelsep=7pt \leftmargin=26pt \topsep=3pt%
\setlength{\listparindent}{\saveparindent}%
\setlength{\parsep}{\saveparskip}%
\setlength{\itemsep}{3pt} }}{\end{list}}

% =========================================================================

% General

\newlength{\savejot}
\setlength{\savejot}{\jot}


\newcommand{\heading}[1]{{\vspace{10pt}\noindent{\textsc{#1}}}}
\newcommand{\noskipheading}[1]{{\noindent{\textsc{#1}}}}

\newcommand{\headingg}[1]{{\textsc{#1}}}
\newcommand{\Heading}[1]{{\vspace{8pt}\noindent\textbf{#1}}}

\newcommand{\emptystring}{\varepsilon}
\newcommand{\concat}{\:\|\:}
\newcommand{\Concat}{\;\|\;}
\newcommand{\Colon}{{\,:\;\,}}

\newcommand{\N}{{{\sf N}}}
\newcommand{\R}{{{\rm\bf R}}}
\newcommand{\Y}{{{\sf Y}}}
\newcommand{\Z}{{{Z}}}

\newcommand{\calC}{{\cal C}}
\newcommand{\calD}{{\cal D}}
\newcommand{\calE}{{\cal E}}
\newcommand{\calF}{{\cal F}}
\newcommand{\calI}{{\cal I}}
\newcommand{\calO}{{\cal O}}
\newcommand{\calP}{{\cal P}}
\newcommand{\calQ}{{\cal Q}}
\newcommand{\calR}{{\cal R}}
\newcommand{\calG}{{\cal G}}
\newcommand{\calK}{{\cal K}}
\newcommand{\calM}{{\cal M}}
\newcommand{\calN}{{\cal N}}
\newcommand{\calT}{{\cal T}}

\newcommand{\D}{{\calD}}
\newcommand{\E}{{\calE}}
\newcommand{\K}{{\calK}}
\newcommand{\I}{{\calI}}
\renewcommand{\R}{{\calR}}
\renewcommand{\O}{{\calO}}
\newcommand{\M}{{\mathsf{Message}}}
\newcommand{\C}{{\mathsf{Ciphertext}}}


\newcommand{\Key}{\mathsf{Key}}
\newcommand{\IVspace}{\mathsf{IV}}
\newcommand{\Nonce}{\mathsf{Nonce}}
\newcommand{\Plaintext}{\mathsf{Plaintext}}
\newcommand{\Ciphertext}{\mathsf{Ciphertext}}

\newcommand{\goesto}{{\rightarrow}}
\newcommand{\eqdef}{\stackrel{\rm def}{=}}
\newcommand{\getsr}{{\:\stackrel{{\scriptscriptstyle \hspace{0.2em}\$}} {\leftarrow}\:}}
\newcommand{\getsd}{{\:\stackrel{{\scriptscriptstyle \hspace{0.2em}D}} {\leftarrow}\:}}
\renewcommand{\choose}[2]{{{#1}\atopwithdelims(){#2}}}
\newcommand{\abs}[1]{{\displaystyle \left| {#1} \right| }}
\newcommand{\EE}[1]{{\E\left[{#1}\right]}}

\newcommand{\Damgard}{{\mbox{Damg{\aa}rd}}}

\newcommand{\strtonum}[1]{{ \mathsf{str2num}\left({#1}\right) }}
\newcommand{\numtostr}[2]{{ \mathsf{num2str}_{\scriptscriptstyle #2} 
                              \left( {#1} \right) }}

\newcommand{\Adv}{{\mathbf{\bf Adv}}}

\newcommand{\Randword}{\mathrm{Rand}}
\newcommand{\Permword}{\mathrm{Perm}}
\newcommand{\Randd}[2]{\ensuremath{\Randword({#1},{#2})}}
\newcommand{\Rand}[1]{\ensuremath{\Randword({#1})}}
\newcommand{\Rn}{{\calR_n}}
\newcommand{\Pn}{{\calP_n}}
\newcommand{\Perm}[1]{\ensuremath{\Permword({#1})}}
\newcommand{\Permm}[2]{\ensuremath{\Permword({#1},{#2})}}

\newcommand{\GF}{{\mathrm{GF}}}

\newcommand{\code}[1]{{\langle{#1}\rangle}}

\newcommand{\then}{{;\;}}
\newcommand{\andthen}{{\::\;\;}}

\newcommand{\ie}{i.e.}
\newcommand{\eg}{e.g.}
\newcommand{\adv}{\ensuremath{\mathrm{Adv}}}
\newcommand{\hdir}{\mbox{\vspace{6mm}\~{ }}}



% ========================================================================

\makeatletter
% from  /usr/local/lib/tex/inputs/article.sty  and
%       /usr/local/lib/tex/inputs/latex.tex

\newcommand{\imply}{\boldmath{\Rightarrow}}

\newcommand{\IV}{{V}}

\newcommand{\CBCexplicitIV} {{\mathrm{CBC }}}
\newcommand{\CBCrandomIV}   {{\mathrm{CBC \$   }}}
\newcommand{\CBCencipherIVone} {{\mathrm{CBC 1}}}
\newcommand{\CBCencipherIVtwo} {{\mathrm{CBC 2}}}
\newcommand{\CBCencipherIVtwox} {{\mathrm{CBC 2 X}}}

\newcommand{\simulate}[3]{{\mbox{\textsl{SimulateOracles}}(#1,#2,#3)}}
\renewcommand{\Y}{{\cal Y}}
\newcommand{\setinvalid}{\mathrm{Invalid}}
\newcommand{\X}{{\cal X}}
\newcommand{\range}{\mathrm{Range}}
\newcommand{\domain}{\mathrm{Domain}}
\newcommand{\ran}{\mathrm{Range}}
\newcommand{\dom}{\domain}
\newcommand{\gamef}{\mathrm{GameF}}
\newcommand{\gamer}{\mathrm{Game\$}}
\newcommand{\cran}{\overline{\range}}
\newcommand{\cdom}{\overline{\domain}}
\newcommand{\bad}{\mathit{bad}}

\newcommand{\PrK}{{\mbox{$\Pr_K$}}}
\newcommand{\PrF}{{\mbox{$\Pr_F$}}}
\newcommand{\Bad}{{\mathsf{BAD}}}

\newcommand{\ctr}{\mathrm{CTR}}
\newcommand{\ctrFE}{{\ctr[F \times E]}}
\newcommand{\ctrRE}{{\ctr[{\cal R}_1 \times E]}}
\newcommand{\ctrRR}{{\ctr[{\cal R}_1 \times {\cal R}_2]}}
\newcommand{\KeyE}{\Key_E}
\newcommand{\KeyF}{\Key_F}
\newcommand{\TimeX}{\mathrm{Time}}
\newcommand{\authh}{{\mathrm{auth}*}}
\newcommand{\LAND}{{\;\wedge\;}}


\newcommand{\gamesfontsize}{\footnotesize}


\newcommand{\soln}[1]{\vspace*{1ex}\fbox{\begin{minipage}[c]{6.0in}{\textbf{\textit{Solution.}}\/#1}\end{minipage}} \vspace*{1ex}}


\newcommand{\mpage}[2]{\begin{minipage}{#1\textwidth}\gamesfontsize #2 \end{minipage}}
\newcommand{\fpage}[2]{\framebox{\begin{minipage}{#1\textwidth}\gamesfontsize #2 \end{minipage}}}

\newcommand{\hfpages}[3]{\hfpagess{#1}{#1}{#2}{#3}}
\newcommand{\hfpagess}[4]{
		\begin{tabular}{c@{\hspace*{.5em}}c}
		\framebox{\begin{minipage}[t]{#1\textwidth}\setstretch{1.1}\gamesfontsize #3 \end{minipage}} 
		&  
		\framebox{\begin{minipage}[t]{#2\textwidth}\setstretch{1.1}\gamesfontsize #4 \end{minipage}}
		\end{tabular}
	}
\newcommand{\hfpagesss}[6]{
		\begin{tabular}{c@{\hspace*{.5em}}c@{\hspace*{.5em}}c}
		\framebox{\begin{minipage}[t]{#1\textwidth}\gamesfontsize #4 \end{minipage}} 
		&  
		\framebox{\begin{minipage}[t]{#2\textwidth}\gamesfontsize #5 \end{minipage}}
		&  
		\framebox{\begin{minipage}[t]{#3\textwidth}\gamesfontsize #6 \end{minipage}}
		\end{tabular}
	}


\def\codestretch{1.1}

\newcommand{\hpagesl}[3]{
	\begin{tabular}{c|c}
	  \begin{minipage}{#1\textwidth}\setstretch{\codestretch} #2 \end{minipage} 
	  & 
	  \begin{minipage}{#1\textwidth} #3 \end{minipage}
	\end{tabular}
	}

\newcommand{\hpagessl}[4]{
	\begin{tabular}{c|c}
	   \begin{minipage}[t]{#1\textwidth}\setstretch{\codestretch} #3 \end{minipage} 
	   & 
	   \begin{minipage}[t]{#2\textwidth}\setstretch{\codestretch} #4 \end{minipage}
	\end{tabular}
	}

\newcommand{\hpages}[3]{
	\begin{tabular}{cc}
	   \begin{minipage}[t]{#1\textwidth}\setstretch{\codestretch} #2 \end{minipage} 
	   & 
	   \begin{minipage}[t]{#1\textwidth}\setstretch{\codestretch} #3 \end{minipage}
	\end{tabular}
	}
\newcommand{\hpagess}[4]{
    \begin{tabular}{cc}
	   \begin{minipage}[t]{#1\textwidth}\setstretch{\codestretch} #3 \end{minipage} 
	   &
	   \begin{minipage}[t]{#2\textwidth}\setstretch{\codestretch} #4 \end{minipage} 
    \end{tabular}
	}

\newcommand{\hpagesss}[6]{
	\begin{tabular}{ccc}
	\begin{minipage}[t]{#1\textwidth}\setstretch{\codestretch} #4 \end{minipage} & 
	\begin{minipage}[t]{#2\textwidth}\setstretch{\codestretch} #5 \end{minipage} &
	\begin{minipage}[t]{#3\textwidth}\setstretch{\codestretch} #6 \end{minipage}
	\end{tabular}}
\newcommand{\hpagesssl}[6]{
	\begin{tabular}{c|c|c}
	\begin{minipage}[t]{#1\textwidth}\setstretch{\codestretch} #4 \end{minipage} & 
	\begin{minipage}[t]{#2\textwidth}\setstretch{\codestretch} #5 \end{minipage} &
	\begin{minipage}[t]{#3\textwidth}\setstretch{\codestretch} #6 \end{minipage}
	\end{tabular}}
\newcommand{\hpagessss}[8]{
	\begin{tabular}{cccc}
	\begin{minipage}[t]{#1\textwidth}\setstretch{\codestretch} #5 \end{minipage} & 
	\begin{minipage}[t]{#2\textwidth}\setstretch{\codestretch} #6 \end{minipage} &
	\begin{minipage}[t]{#3\textwidth}\setstretch{\codestretch} #7 \end{minipage}
	\begin{minipage}[t]{#4\textwidth}\setstretch{\codestretch} #8 \end{minipage}
	\end{tabular}}


\newcommand{\query}[1]{\procfont{query} {#1}:}
\newcommand{\queryl}[1]{\underline{\procfont{query} {#1}:}}
\newcommand{\oracle}[1]{\underline{\procfont{oracle} {#1}:}}
\newcommand{\oraclev}[1]{\underline{\procfont{oracle} {#1}:}\smallskip}
\newcommand{\procedure}[1]{\underline{\procfont{procedure} {#1}:}}
\newcommand{\procedurev}[1]{\underline{\procfont{procedure} {#1}:}\smallskip}
\newcommand{\subroutine}[1]{\underline{\procfont{subroutine} {#1}:}}
\newcommand{\subroutinev}[1]{\underline{\procfont{subroutine} {#1}:}\smallskip}
\newcommand{\subroutinenl}[1]{{\procfont{subroutine} {#1}:}}
\newcommand{\subroutinenlv}[1]{{\procfont{subroutine} {#1}:}\smallskip}
\newcommand{\adversary}[1]{\underline{\procfont{adversary} {#1}:}}
\newcommand{\adversaryv}[1]{\underline{\procfont{adversary} {#1}:}\smallskip}
\newcommand{\experiment}[1]{\underline{{#1}}}
\newcommand{\experimentv}[1]{\underline{{#1}}\smallskip}


\usepackage{algorithm}
\usepackage{caption}
\usepackage{comment}
\usepackage{dirtytalk}
\usepackage{algpseudocode}
\usepackage{url}
\usepackage{todonotes}
\newcommand{\bitsoracle}{{\mathsf{Bits}}}
\newcommand{\indcpa}{{\mathrm{IND}\mbox{-}\mathrm{CPA}}}
\newcommand{\indrcpa}{{\mathrm{IND}\$\mbox{-}\mathrm{CPA}}}
\newcommand{\Func}{\mathsf{Func}}
%\newcommand{\Perm}{\mathsf{Perm}}
\newcommand{\Prob}[1]{\Pr\left[#1\right]}
\newcommand{\calX}{\mathcal{X}}
\newcommand{\thatis}{i.e. }
\newcommand{\forexample}{e.g. }
\def\NoNumber#1{{\def\alglinenumber##1{}\State #1}\addtocounter{ALG@line}{-1}}

\DeclareCaptionFormat{myformat}{#3}
\captionsetup[algorithm]{format=myformat}
 
%\newcommand{\Adv}{\mathrm{Adv}}
\newcommand{\ExpPRF}[2]{\mathsf{Exp}^{\mathrm{prf}}_{#1}{(#2)}}
\newcommand{\AdvPRF}[2]{\Adv^{\mathrm{prf}}_{#1}(#2)}
\newcommand{\ExpVP}[2]{\mathsf{Exp}^{\mathrm{vp}}_{#1}{(#2)}}
\newcommand{\AdvVP}[2]{\Adv^{\mathrm{vp}}_{#1}(#2)}
\newcommand{\ExpPRP}[2]{\mathsf{Exp}^{\mathrm{prp}}_{#1}{(#2)}}
\newcommand{\AdvPRP}[2]{\Adv^{\mathrm{prp}}_{#1}(#2)}
\newcommand{\ExpGuess}[2]{\mathsf{Exp}^{\mathrm{guess}}_{#1}{(#2)}}
\newcommand{\AdvGuess}[2]{\Adv^{\mathrm{guess}}_{#1}(#2)}

\newcommand{\calV}{\mathcal{V}}

%%%%%%%%%%%%%%%%%%%%%%%%%%%%%%%%%%%%%%%%%%%%%%%%%%%%%%%%
\title{{\bf  Introduction to Modern Cryptography}\\ Problem Set 1 \\[2ex] }
\date{}
\author{}
\begin{document}
\maketitle
%%%%%%%%%%%%%%%%%%%%%%%%%%%%%%%%%%%%%%%%%%%%%%%%%%%%%%%%
\thispagestyle{empty}

\paragraph{Due date.} Jan 31, 2022 (by 11:59pm).  Please turn in a PDF
document, and please try to typeset it in LaTeX.  (I'll provide the
source for this assignment, so you have a starting point.)
Since you can work in groups, please put all names on your solution
set, and please include the last name of at least one of your group members in the filename.

\paragraph{Notational reminders.}  Recall that when $X,Y$ are
bitstrings, $X \concat Y$ is the concatenation of the two.   When
$|X|=|Y|$, we write $X \xor Y$ for their bitwise exclusive-or.

\paragraph{Problem 1.}
 
\begin{figure*}[ht]
\center
\fpage{.7}{
\hpagess{.6}{.45}
{
\experimentv{$\ExpGuess{F,\tau}{A}$:}\\
$\calX \gets \emptyset$\\
$K \getsr \calK$\\
$(P,X')\getsr A^{\calO(\cdot)}$\\
if $\mathsf{LSB}_\tau(F_K(X'))=P$ and $X' \notin \calX$ then Ret 1 \\
Ret 0
}
{
\oraclev{$\calO(X)$}\\
$\calX \gets \calX \cup \{X\}$\\
Ret $F_K(X)$
}
}
%
\fpage{.7}{
\hpagess{.6}{.45}
{
\experimentv{$\ExpPRF{F}{A}$:}\\
$b \getsr \bits$\\
$K \getsr \calK$\\
$\rho \getsr \Func(\calD,\calR)$\\
$b'\getsr A^{\calO(\cdot)}$\\
Ret $[b'=b]$
}
{
\oraclev{$\calO(X)$}\\
if $b=1$ then $Y \gets F_K(X)$\\
else $Y \gets \rho(X)$\\
Ret $Y$
}
}
\caption{({\bf Top.}) The ``guess the LSBs'' game for a function family
  $F\colon \calK \times \calD \to \calR$ where $\calR \subseteq
  \bits^*$.  On input a
  bitstring~$X=x_1x_2\cdots x_t$ the
  function $\mathsf{LSB_\tau}(X)=x_1 x_2 \cdots 
  x_{\min\{t,\tau\}}$.  ({\bf Bottom.})  The psuedorandom function security notion.}
\label{fig:guess}
\end{figure*}

Fix non-empty sets $\calK,\calD,\calR$ where $\calR \subseteq
\bits^*$, and let $F \colon \calK \times
\calD \to \calR$ be a function family (as usual, indexed by
$K\in\calK$).  In Figure~\ref{fig:guess} we define a new (toy) security
notion called the ``guess-the-LSBs'' notion.  In it, the adversary is
given access to an oracle for $F_K(\cdot)$ for a uniform, secret
key~$K$.  At the end of its execution, the adversary must output a
pair $(P,X')$ and wins the game if the $\tau$ least-significant bits
of $F_K(X')$ are~$P$.
Note that the
adversary does not get credit for outputting a pair $(P,X')$ such
that $X'$ was previously asked to its oracle; that would be cheating!
You can assume without loss that there are no pointless queries ---~in this case, no repeated
queries (because $F_K$ is deterministic) and no queries $X \notin
\calD$~--- and that $X' \in \calD$.

The $\tau$-LSB-guessing advantage of an adversary~$A$, against
function family~$F$, is
$\AdvGuess{F,\tau}{A}=\Prob{\ExpGuess{F,\tau}{A}=1}$.  
In the experiment we track the following adversarial resources: time
complexity~$t$, number of queries asked~$q$, and~$\mu$ the total
bitlength of all queries plus the length of the adversary's output.

({\bf Part a.}) Please fill in the following theorem, and provide a nice proof of it.

\begin{theorem}
Fix integers $N,\tau$ such that $0 < \tau \leq N$.
Fix non-empty sets $\calK,\calD$ and let $F \colon \calK \times
\calD \to \bits^N$ be a function family.  Let~$A$ be an adversary
using resources~$t,q,\mu$.  Then there exists a PRF-adversary~$B$
(explicitly constructed in the proof of this theorem) such that
\[
\AdvGuess{F,\tau}{A} \leq 1 \AdvPRF{F}{B} + \frac{1}{{{(2^\tau)}^2}^n} \mbox{   [fill in
  appropriate (small) constants $c_1,c_2$] }
\]
where $B$ has time complexity~$t'+q$,
asks $q'=q$ queries, these of total length
$\mu'=\mu$%[\mbox{FILL IN}]$.
\end{theorem}

\textbf{Part a.}

We will use $P_r[\ExpPRF{}{B} = 1]$ and condition this probability by the challenge bit $b$ for \ExpPRF{}{B}^{\calO(\cdot)}$.We need to construct adversary $B$ to simulate adversary~$A$.

\begin{comment}
\adversaryv{$B^{\calO}$}\\
(Y', X')$\leftarrow$ {$A^{\calO(\cdot)}$}\\
If $\calO (X')=Y'$ then Return 1\\
Return 0
\end{comment}

\begin{algorithm}
\renewcommand{\thealgorithm}{}
\caption{\textbf{adversary $B^{\calO}$}}
\label{alg:cap}
\begin{algorithmic}

\State $(Y', X') \gets A^{\calO(\cdot)}$
\If{$\calO (X')=Y'$}
\State \Return 1
\EndIf
\State \Return 0
\end{algorithmic}
\end{algorithm}

When challenge bit $b=1$, adversary $A$ expects $F_K$ for a random key, which perfectly simulates $A$'s queries in $F_k$ or when challenge bit $b=0$, adversary $A$ expects random function $\rho$. Thus, $P_r[\ExpPRF{}{B} = 1 \;|\; b=0] = P_r[\ExpGuess{}{A} = 1 \;|\; b=0]$.


By construction, B has time complexity of $t'+q$, asks $q'=q$ queries, those total length $\mu'=\mu$. 

\vspace{0.5cm}

\underline{$b=1 \; so \; \calO = F_K$}

$P_r\left[\ExpPRF{}{B} = 1 \;|\; b=1\right] $

$=P_r\left[F_K(X') = Y' \right]  $

$=P_r[\ExpGuess{}{A} = 1 \;|\; b=1]$

($it \;is\; function\; family \;whose\; security \;is \;being \;examined$)

\vspace{0.5cm}

\underline{$b=1 \; so \; \calO = \rho$}

$P_r[\ExpPRF{}{B} = 1 \;|\; b=0] $

$= 1-P_r[\ExpPRF{}{B} = 0 \;|\; b=0]$

$= 1-P_r[\rho(X') = Y'] $

$= 1-\left[\frac{1}{{{(2^\tau)}^2}^m} \right]$


$\left(b=0 : \; |\Func(m,n)| = \frac{1}{{{(2^\tau)}^2}^m}$ because $|\Func(D,R)| = \frac{1}{{{(2^R)}^2}^D}}, $ F \colon \calK \times
\calD \to \bits^N$\right$

\vspace{0.5cm}

Hence, \\$P_r[\ExpPRF{}{B} = 1] = P_r[\ExpPRF{}{B} = 1 \;|\; b=1] 
+ P_r[\ExpPRF{}{B} = 1 \;|\; b=0]
$

$\AdvPRF{F}{B} = $\\
$=2\left[\frac{1}{2}(P_r(\ExpGuess{F}{A} =1)+
P_r(\ExpPRF{}{B} =1 \; | \; b=0)
)\right] -1$


$\left( \AdvPRF{F}{B} = 2\cdot P_r\left(\ExpPRF{F}{B} = 1\right) - 1 \right)$

\vspace{0.5cm}

$\AdvPRF{F}{B} $\\
$=2\left[\frac{1}{2}\left((P_r(\ExpGuess{F,\tau}{A} =1) + \left[1 -
\frac{1}{{{(2^\tau)}^2}^m}\right]\right)  \right]- 1
= 
\left P_r[\ExpGuess{F,\tau}{A}=1] - \frac{1}{{{(2^\tau)}^2}^m}}
$

Hence, $\AdvGuess{F,\tau}{A} \leq \AdvPRF{F}{B}+\left[\frac{1}{{{(2^\tau)}^2}^m}\right]
$

Since $c_2$ is \say{very small} we conclude that F is a secure $PRF$ and it follows that \say{Guess the LSBs} is secure, or difficult to guess LSBs.

\vspace{0.5cm}



({\bf Part b.})  The security bound that you proved in Part a should be a
function of the parameter~$\tau$.  Consider the full range of values
for $\tau$: When is the theorem most ``useful''.  What is the
relationship between the additive term~$c_2$ that you find, and the
``trivial'' winning probability?


\textbf{Part b.}

$\left|\AdvGuess{F,\tau}{A}-\AdvPRF{F}{B}\right| \leq \frac{1}{{{(2^\tau)}^2}^n}$

We can expect the behaviour of the oracle. When $\AdvGuess{F, \tau}{A}$ is close to 0, then the oracle behaves random function. 

Also, $P$ is a function of parameter $\tau$ and the additive term $c_2$ gets \say{tiny} when $\tau$ is a large number as its range is $2^\tau$ for all $0\leq \tau \leq N$ so the winning probability gets \say{tiny}.


%%%%%%%%%%%%%%%%%%%%%%%%%%%%%%%%%%%%%%%%%%%%%%%

\paragraph{Problem 2.}
Fix $k,n>0$ and let $E \colon \bits^k \times \bits^n \to \bits^n$ be a
blockcipher.  We would like to construct, from~$E$ an new blockcipher
that is ``wider''; in particular $F \colon \bits^k \times \bits^{2n}
\to \bits^{2n}$.  (Why build from~$E$, instead of building~$F$ from
scratch?  Maybe~$E$ is AES, or some other standardized blockcipher,
that has undergone years of cryptanalysis.  Better to build from that,
than to try to design your own $2n$-bit blockcipher.)  Here's the
candidate construction:  for every $K \in \bits^k, X \in \bits^{2n}$
let $F_K(X) = E_K(X_L) || E_K(X_R)$, where $X = X_L \concat X_R$.
(You can infer from the description of~$E$ that $|X_L|=|X_R|=n$.)

\begin{itemize}
\item Verify that~$F$ is, indeed, a blockcipher.
\end{itemize}

\paragraph{Proof F is a Blockcipher}
We know by construction that $E \colon \bits^k \times \bits^n \to \bits^n$ is indeed a blockcipher. Therefore to prove that $F \colon \bits^k \times \bits^{2n}
\to \bits^{2n}$ is also a Blockcipher, we will analyze the implementation of $F$. By hypothesis,  $X = X_L \concat X_R$. Begin by letting $X_L = L$ with $L \getsr \bits^n$ and $X_R \in \bits^n$. Since $E$ is a Blockcipher, we know that $E_k(X_R)$ will be a Bijection and so we will have a unique set of solutions to $F_k(X) = E_k({L}) || E_k(X_R)$ for all $X_R \in \bits^{n}$. Recall that $E_k({L})$ is constant for some $L$. Now, consider the case of $X_L = M$ with $M \getsr \bits^n - {L}$. We know again that $E_k(M) \neq E_k(L)$ using the fact that $E_k$ is Bijective. Hence for all $X_R \in \bits^n$ we know that $F_k(X) = E_k(M) || E_k(X_R)$ is also a unique set of solutions in terms of $X_L$ and $X_R$. Therefore, $F$ is  a Blockcipher.

\begin{itemize}
\item Show that~$F$ \emph{is not} a secure PRP, even if~$E$
  \emph{is}.  Specifically, give an attack ---~described as an
  adversary~$A$ and using the style and notation that we've seen in
  the videos~--- that uses just a few oracle queries, and not much
  running time, yet gains PRP-advantage against~$F$ that is close to
  1.  Please give a step-by-step analysis of the advantage, i.e.,
  carefully reason out an expression for~$\AdvPRP{F}{A}$, where~$A$ is
  \emph{your} adversary/attack. 
\end{itemize}

\paragraph{Proof F is NOT Secure PRP}\quad
\\


\adversaryv{$A^{\calO(.)}_F$}\\
Query $\calO(0^n || 0^n)$\\
Call the Result $Y_L || Y_R$\\
If $Y_L == Y_R$ then Return 1\\
Return 0\\

$\AdvPRP{F}{A} = 2*[Pr(\ExpPRP{F}{A} = 1 | b = 1)Pr(b = 1) +  Pr(\ExpPRP{F}{A} = 1 | b = 0)Pr(b = 0)] - 1$\\

Because $Pr(b = 1) = \frac{1}{2}$ and $Pr(b = 0) = 1 - \frac{1}{2} = \frac{1}{2}$ \\
\\
We can simplify \\
\\
$\AdvPRP{F}{A} = 2*[\frac{1}{2}Pr(\ExpPRP{F}{A} = 1 | b = 1) +  \frac{1}{2}Pr(\ExpPRP{F}{A} = 1 | b = 0)] - 1$\\
\\
Additionally, note that $Pr(\ExpPRP{F}{A} = 1 | b = 0) = 1 - Pr(\ExpPRP{F}{A} = 0 | b = 0)$\\


Then $\AdvPRP{F}{A} = 2*[\frac{1}{2}Pr(\ExpPRP{F}{A} = 1 | b = 1) +  \frac{1}{2}(1 - Pr(\ExpPRP{F}{A} = 0 | b = 0))] - 1$\\

Which simplifies to $\AdvPRP{F}{A} = Pr(\ExpPRP{F}{A} = 1 | b = 1) +  1 - Pr(\ExpPRP{F}{A} = 0 | b = 0) - 1$ = \\

$\AdvPRP{F}{A} = Pr(\ExpPRP{F}{A} = 1 | b = 1) - Pr(\ExpPRP{F}{A} = 0 | b = 0)$\\

By analyzing $Pr(\ExpPRP{F}{A} = 1 | b = 1)$ we can see that we will be testing our experiment against $F_K$ and we know that for $F_K(0^n|| 0^n)$ our adversary will output the correct result 1 by construction.\\ 
\\
Hence $Pr(\ExpPRP{F}{A} = 1 | b = 1) = 1$\\

Leaving us with $\AdvPRP{F}{A} = 1 - Pr(\ExpPRP{F}{A} = 0 | b = 0)$\\

Now, consider the term $Pr(\ExpPRP{F}{A} = 0 | b = 0)$\\
\\
We know that our Attacker will be querying against a random permutation since $b = 0$
\\

So, we need to consider the possibility that our experiment outputs a zero, which only happens if our Attack returns 1 - meaning our guess is incorrect. \\
\\
Because out attack calls $\calO(0^n || 0^n)$ to a random permutation, we need to analyze the probability that some output is symmetric.\\
\\
Since we have an output of the form $\bits^{2n}$ we know that given any left-hand side $\bits^{n}$ the probability that the other side will be the exact same for a permutation is $\frac{1}{2^n}$\\
\\
Hence, $Pr(\ExpPRP{F}{A} = 0 | b = 0) = \frac{1}{2^n}$\\
\\
$\AdvPRP{F}{A} = 1 - \frac{1}{2^n} \approx 1$ and we can see that $F$ is not secure PRP.


%%%%%%%%%%%%%%%%%%%%%%%%%%%%%%%%%%%%%%%%%%%

\paragraph{Problem 3.}
Let $E:\bits^k \times \bits^n \to \bits^n$ be a blockcipher.  From it,
construct $E^{(2)}\colon \bits^{2k} \times \bits^n \to \bits^n$ by
$E^{(2)}_{K1,K2}(X) = E_{K2}(E_{K1}(X))$. Prove that if $E$~is a
secure PRP, then so is~$E^{(2)}$.  

To get you started, here's a reminder of how to approach this
proof. Given an adversary~$A$ that attacks the PRP-security of
$E^{(2)}$, build an adversary~$B$ that attacks the PRP-security
of~$E$.  Carefully show that the PRP-advantage of~$B$ upperbounds the
PRP-advantage of~$A$, by analyzing the probability that~$B$ ``wins''
its experiment.  When building your adversary~$B$, try to make it as
simple as possible, and as parsimonious as possible with respect to
its resources.  

Finally, write a nice theorem statement to encapsulate
your result. 

\vspace{1cm}


\end{document}

